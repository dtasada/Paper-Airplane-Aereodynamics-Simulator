\documentclass[a4paper,12pt]{report}

% Packages
\usepackage{amsmath}
\usepackage{amssymb}
\usepackage{bibentry}
\usepackage{geometry}
\usepackage{graphicx}
\usepackage{hyphenat}
\usepackage{indentfirst}
\usepackage{titlesec}
\usepackage{url}
\usepackage[normalem]{ulem}
\usepackage[backend=biber, style=apa]{biblatex}
\usepackage[
colorlinks=true,
linkcolor=blue,
urlcolor=cyan,
citecolor=blue,
]{hyperref} % For hyperlinks

\addbibresource{references.bib}
\geometry{margin=1in}
\titleformat{\chapter}[display]
{\normalfont\huge\bfseries}{\chaptertitlename\ \thechapter}{20pt}{\Huge}
\titlespacing{\chapter}{0pt}{0pt}{0pt}

% Title and Author
\title{Computational Fluid Dynamics With a Paper Airplane}
\author{
  Tasada, Daniel\\
  \and
  Tse, Nathan\\
}
\date{\today}

\begin{document}

% Title and abstract
\maketitle
\begin{abstract}
	In this paper, we investigate the relationship between an airplane's shape and its performance.
	Our results show that \dots
\end{abstract}

% Table of Contents
\tableofcontents

% Introduction
\chapter{Introduction}
This thesis covers the simulation of the aerodynamics of an airplane, using
our own Computation Fluid Dynamics (or CFD) model.

The goal is to calculate the ideal shape of an airplane for its aerodynamic performance.
We aim to do this by creating a Machine Learning model, and feed it the results of the CFD,
which should result in our final model.

The thesis questions are the following:
\begin{itemize}
	\item{How do the different aspects of fluid dynamics work and how do we implement it in a computer program?}
	\item{How do we dynamically generate 3D models?}
	\item{How does an airplane's wing shape influence its performance?}
\end{itemize}

% Methodology
\chapter{Execution}
\section{Methodology}
To achieve our goal, we've had first had to learn about particle dynamics.
It took a while to get an understanding, but here's what we found:

\subsection{The Math}
\subsubsection{Particle collision}
Part one of the fluid dynamics simulation is particle collision.
Fluid dynamics, in essence is particle physics, where every air molecule is a "particle".
We start with a container, and a bunch of air molecules, all of which interact with
each other to create a fluid. This interaction is effectively the collisions between particles.
It turns out that simulating  particle dynamics in two dimensions is pretty straightforward.
But when you throw in that Z-axis, it gets a lot harder.
The main article I used for this is \hyperlink{http://www.hakenberg.de/diffgeo/collision_resolution.htm}{Rigid Body Collision Resolution} (\cite{hakenberg}).

Here is a set of formulas necessary to perform the calculations that describe
the behavior of a pair particle before and after their collision:

The following variables are necessary to perform the calculations:
\[
\begin{array}{ll}
    \text{Inertia tensor $I$ } (kg\cdot m^2): & L = \frac{L}{\omega}; \\
    \text{Angular momentum $L$ } (kg\cdot m^2/s): & L = mvr; \\
    \text{Angular velocity $\omega$ } (rad/s): & \omega = \frac{\Delta \theta}{\Delta t}; \\

	\\

	\text{Normal of contact } (n \in \mathbb{R}^3) \text{ in world coordinates away from body 1}; \\
	\text{Point of contact } (r_i \in \mathbb{R}^3) \text{ in world coordinates with respect to $p_i$}; \\
\end{array}
\]

Where $i$ represents one of two particles in a given collision:
\[
\begin{array}{ll}
	\text{Velocity after collision} & \tilde{v}_i, \\ 
	\text{Angular velocity after collision} & \tilde{\omega}_i, \\
	\text{Constant} & \lambda, \\
\end{array}
\]

The following formulas represent the relation between particles:
\[
\begin{array}{cc}
	\tilde{v}_1 = v_1 - \frac{\lambda}{m_1} n; \\ 
	\tilde{v}_2 = v_2 + \frac{\lambda}{m_2} n; \\
	\tilde{\omega}_1 = \omega_1 - \Delta q_1; \\
	\tilde{\omega}_2 = \omega_2 + \Delta q_2; \\

	\text{where } q_i := I_i^{-1} \cdot R_i^{-1} \cdot (r_i\times n), \\
	\text{and } \lambda = 2 \frac{n v_1 - n v_2 + \omega_1 I_1 q_1 - \omega_2 I_2 q_2}
	{(\frac{1}{m_1} + \frac{1}{m_2})n^2 + q_1 I_1 q_1 + q_2 I_2} \\
\end{array}
\]

\subsection{The Code}
TODO: Explain container system, maybe show code samples

% Results
\chapter{Results}
Present your experimental or theoretical results in this section. Use tables and figures to illustrate important points.

% Conclusion
\section{Conclusion}
Summarize the main findings of the paper. Mention potential future work or research directions.

% References
\nocite{*}
\printbibliography

\end{document}
